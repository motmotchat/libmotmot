\documentclass[12pt]{article}

\usepackage{fancyhdr, geometry, hyperref, xspace}
\usepackage[parfill]{parskip}

\geometry{letterpaper,text={7in,9in},centering}
\pagestyle{fancy}

\fancyhead[RO]{Carl Jackson and Max Wang}
\fancyhead[LO]{\url{http://github.com/motmotchat/trill}}

\newcommand{\Trill}{\textsc{Trill}\xspace}

\title{\Trill: Secure firewall-traversing networking}
\author{Carl Jackson and Max Wang}
\date{December 14, 2012}

\makeatletter
\renewcommand{\maketitle}{%
  \thispagestyle{plain}%
  \begin{center}%
    {\LARGE \@title \par}%
    {\large \@author \par}%
    {\large \@date \par}%
  \end{center}%
}
\makeatother

\begin{document}

\maketitle

\section*{Overview}
We implemented \Trill, a networking abstraction that provides secure
communication between federated identities, either of which could be behind a
firewall.

\section*{Security}
The security of our system is based on standard cryptographic infrastructure and
libraries. By ``identities,'' we mean public key pairs, the private half of
which is uniquely known by an individual. This identity can be given one or more
names by means of X.509 certificates, and standard mechanisms like TLS allow
owners of identities to securely establish a communication channel and mutually
verify the other's identity.

\section*{Federated Identity}
As with other federated protocols like XMPP and email, \Trill handles are based
on DNS. In particular, \Trill's handles use RFC 2822's \verb`addr-spec`
specification, in which every \verb`domain` is allowed complete control over the
\Trill handles given out as \verb`local-part`'s at their domain. Domains wishing
to run \Trill must also publish information about its \Trill servers using DNS
SRV records.

Each \Trill domain hosts a federation server, which is able to delegate
authority for identities to various clients.  Any client may present the server
with an identity and a password for that identity.  If the client's
authentication succeeds, the server signs the client's key and returns the
resultant certificate.  Using these certificates, clients can then prove their
handles securely to one another without further use of the server.

Note that the assigning of handles is orthogonal to the security of the protocol
itself, and is only useful in associating human-friendly names to their actual
cryptographic identity, i.e., their private key. Users who do not wish to use
\Trill's federated identity system can instead self-sign their certificates and
verify these credentials out-of-band. This is very similar to PGP's ``web of
trust,'' and while we have not implemented such a system, we imagine the
integration of \Trill with PGP would be trivial. An SSH-like Trust On First Use
(TOFU) system might also be viable here.

\section*{Establishing Trust}
% Trust can be rooted two ways:
% - Traditional CAs: Verify plume certificate against CAs, verify identities
%   against plume certificate
% - DNSSEC: the key (or a fingerprint of the key) for any domain can be added as
%   a signed TXT record. We can root our trust back up to the DNSSEC root.
% This second way seems more secure: root CAs are allowed to sign any record,
% compromising some CAs does not seem difficult. DNS has the interesting
% property of hierarchically namespaced trust: it is likely that a root zone
% compromise will be difficult (root keys only used rarely) and discovered
% quickly; owners of ".cn" keys will be unable to spoof ".com"s, etc.

% Also semantics of plume CAs are "have control of this domain" => "can give out
% identities at this domain." No communications security is based on this, only
% naming.

\section*{Firewall Traversal}
Due to restrictions on many firewalls and NATs in place on the Internet, it is
impossible for arbitrary clients on the internet to establish peer-to-peer
communication. There are techniques for firewall/NAT traversal which involve
``tricking'' the networking layer into establishing a connection. We implemented
something similar to \texttt{chownat} (\url{http://samy.pl/chownat/}) in \Trill,
which appears to be a standard way to do this.

\section*{Implementation}
% trill - client library. allows plug-and-play event loops, architected to (one
%   day!) support multiple TLS libraries.
% plume - server. separate login and protocol servers. access to domain root
%   keys has very small surface area. chroot more!


\end{document}
