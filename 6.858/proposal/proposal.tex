\documentclass[12pt]{article}

\usepackage{fancyhdr, geometry, xspace}
\usepackage[parfill]{parskip}
\geometry{letterpaper,text={7in,9in},centering}
\pagestyle{fancy}

\newcommand{\Trill}{\textsc{Trill}\xspace}

\title{\Trill: Secure firewall-traversing networking}
\author{Carl Jackson and Max Wang}
\date{November 5, 2012}

\makeatletter
\renewcommand{\maketitle}{%
  \thispagestyle{plain}%
  \begin{center}%
    {\LARGE \@title \par}%
    {\large \@author \par}%
    {\large \@date \par}%
  \end{center}%
}
\makeatother

\begin{document}

\maketitle

\section*{Overview}
We propose to design and implement \Trill, a networking abstraction that
provides secure communication between federated identities, either of which
could be behind a firewall.

\section*{Security}
The security of our system will be based on standard cryptographic
infrastructure and libraries. By ``identities,'' we mean public key pairs, the
private half of which is uniquely known by an individual. This identity can be
given one or more names by means of X.509 certificates, and standard mechanisms
like TLS allow owners of names to securely establish a communication channel
and mutually verify the other's identity. Due to implementation restrictions in
our target environment, we will be using DTLS, a younger protocol that uses UDP
as its transport layer, and we will be using the implementation of DTLS
provided by the open source GnuTLS library.

\section*{Federated Identity}
As with other federated protocols like XMPP and email, \Trill identities will
be based on DNS.  Every domain name with a SRV record can act as a \Trill
identity realm, supporting email address-like identities (an \verb!addr-spec!
as defined in RFC 2822).

To grant a user an identity on a realm, the user first presents a username and
password to the server, after which the server delegates authority for that
name to the client by signing that client's key. This key-signing process is
the method by which peers can authenticate securely to each other without a
centralized broker.

\section*{Firewall Traversal}
% TODO

\section*{Networking Abstractions}
% TODO

\end{document}
