\documentclass[12pt]{article}

\usepackage{fancyhdr, geometry, xspace}
\usepackage[parfill]{parskip}
\geometry{letterpaper,text={7in,9in},centering}
\pagestyle{fancy}

\newcommand{\Trill}{\textsc{Trill}\xspace}

\title{\Trill: Secure firewall-traversing networking}
\author{Carl Jackson and Max Wang}
\date{November 5, 2012}

\makeatletter
\renewcommand{\maketitle}{%
  \thispagestyle{plain}%
  \begin{center}%
    {\LARGE \@title \par}%
    {\large \@author \par}%
    {\large \@date \par}%
  \end{center}%
}
\makeatother

\begin{document}

\maketitle

\section*{Overview}
We propose to design and implement \Trill, a networking abstraction that
provides secure communication between federated identities, either of which
could be behind a firewall.

\section*{Security}
The security of our system will be based on standard cryptographic
infrastructure and libraries. By ``identities,'' we mean public key pairs, the
private half of which is uniquely known by an individual. This identity can be
given one or more names by means of X.509 certificates, and standard mechanisms
like TLS allow owners of names to securely establish a communication channel
and mutually verify the other's identity. Due to implementation restrictions in
our target environment, we will be using DTLS, a younger protocol that uses UDP
as its transport layer, and we will be using the implementation of DTLS
provided by the open source GnuTLS library.

\section*{Federated Identity}
As with other federated protocols like XMPP and email, \Trill identities will
be based on DNS.  Every domain name with a SRV record can act as a \Trill
identity realm, supporting email address-like identities (an \verb!addr-spec!
as defined in RFC 2822).

Each \Trill domain should host a federation server, which is able to delegate
authority for identities to various clients.  Any client may present the server
with an identity and a password for that identity.  If the client's
authentication succeeds, the server signs the client's key and returns the
resultant certificate.  Using these certificates, clients can then authenticate
securely to one another without further use of the server.

\section*{Firewall Traversal}
Due to restrictions on many firewalls and NATs in place on the Internet, it is
impossible for arbitrary clients on the internet to establish peer-to-peer
communication. There are techniques for firewall/NAT traversal which involve
``tricking'' the networking layer into establishing a connection. We plan on
reimplementing something similar to \texttt{chownat} (http://samy.pl/chownat/),
which appears to be a standard way to do this.

\section*{Networking Abstractions}
% TODO

\end{document}
